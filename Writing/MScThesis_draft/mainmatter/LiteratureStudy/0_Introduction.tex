This chapter reports the outcome of the literature study aimed at establishing a foundation for developing a robust and quickly adaptive data-driven control solution. Furthermore, it will survey works in this field to highlight the unique features of our work and its novelties. State-of-the-art approaches are explained and evaluated, with the chosen approaches justified.

Reusable rockets and autonomous planetary landing have caused a demand for sophisticated control systems that can quickly adapt to uncertain environments. Lossless convexification has demonstrated success for high-level trajectory planning but has a high computation. Classical controllers, optimal, linear and Lyapunov, used for the mid-level guidance loop, rely on accurate models and can be complex to understand but can provide stability and robustness guarantees under a validated model. Data-driven approaches offer a potential alternative through learning optimal policies via environment interaction, with proven success in working on complex, uncertain and non-linear systems.

This section chapter covers four main topics, first \autoref{sec:landing_control} reviews current GNC systems present in both the academic and industry Worlds, along with the flight phases encountered for a launch vehicle with a reusable first stage, and finally methods used to optimise the landing burn's trajectory. Following this, \autoref{sec:ICS} reviews the main choices in selecting the data-driven strategy for trajectory optimisation. Allowing for a more targetted approach in \autoref{sec:RL} to cover Reinforcement Learning for under the choices made in \autoref{sec:ICS}, culminating in an algorithm selection. Finally, \autoref{sec:RocketModel} first defines the staging and sizing procedure to find a feasible rocket for first stage landing, before building a rocket simulation model.